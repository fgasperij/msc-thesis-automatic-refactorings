section{Justificación}
\subsection{¿Por qué refactorings automáticos de alto nivel?}
% Preguntas guía
%   - Cuál es el potencial de los refactorings automáticos de alto nivel
%   - Por qué creemos que son una buena herramienta

% Por qué refactorings
Los refactorings son una herramienta esencial del arsenal de los ingenieros de
software para minimzar el incremento de la complejidad del software y así
preservar su mantenibilidad. Sin embargo, los beneficios de los refactorings
conllevan riesgos y costos. Veamos cómo las herramientas que automatizan parte
de su realización ayudan a reducirlos:

\begin{description}
    \item[Introducir bugs] Los refactorings automáticos permiten reducir el
      riesgo de introducir bugs porque las herramientas que los aplican
      recibieron más tiempo de desarrollo, testeo y consideración que la que
      cualquier desarrollador puede dedicar a la hora de realizar un refactoring
      manual. Además, cuentan con varias ventajas sobre la aplicación manual:
        \begin{itemize}
            \item cuanto más se usan mejores se vuelven: los errores que encuentren los
            usuarios se pueden arreglar para converger a través del tiempo a que
            se preserve el comportamiento con una probabilidad altísima. Los
            programadores pueden cometer el mismo error repetidas veces; algo que
            contribuye a esto es que la experiencia de los programadores trabajando en
            el proyecto puede variar y el tiempo que hayan pasado trabajando en ese
            proyecto en particular también.
            \item no son afectadas por el contexto: los refactorings manuales son
            realizados por desarrolladores a los cuales los puede afectar el
            cansancio, el estrés, el apuro, etc. factores que aumentan la probabilidad
            de equivocación u olvido en un tarea tediosa y demandante de atención al
            detalle como ésta.
        \end{itemize}
    \item[Tiempo de desarrollo] La aplicación de refactorings es parte de el
      tiempo de desarrollo de un proyecto de software. Al utilizar herramientas
      automáticas para aplicarlos el tiempo invertido en realizarlos se ve
      disminuido porque una herramienta automática es más veloz que un humano,
      si esto no fuera así la herramienta es inútil. Esto representa una gran
      ganancia ya que el mantenimiento de un sistema es su mayor costo, si este
      costo se disminuye se abarataría la producción de software, y,
      considerando la omnipresencia actual del mismo, la herramienta tendría un
      impacto enorme en la sociedad.
    \item[Reducir la calidad del sistema] Toda modificación o extensión tiene el
      riesgo de reducir la calidad del sistema, incluso aquellas que no
      modifican el comportamiento del sistema, como es el caso de los
      refactorings. Al no contar con métricas definitivas para evaluar la
      calidad de un sistema no es simple ver si efectivamente los refactorings
      automáticos mitigarían el riesgo de bajar la calidad del sistema. Sin
      embargo, existen consideraciones que nos llevan a intuir que este puede
      ser el caso.
      % REF al que decía que con refactorings de patrones iban a poder probar más diseños
      UNA REF postula que si los desarrolladores contaran con buenas herramientas
      automáticas para realizar refactorings de alto nivel probarían más opciones de
      diseño, lo cual les permitiría llegar a mejores diseños. No es concluyente, pero
      si ciertas herramientas aportan a que los desarrolladores lleguen a mejores
      diseños tienen una buena chance de aumentar la calidad de los sistemas.
\end{itemize}

Los refactorings automáticos de alto nivel cuentan con el potencial de maximizar
estos beneficios porque el scope de automatización sería mayor. Al reemplazar
más trabajo manual consigue que se ahorre más tiempo de desarrollo y reduce aún
más la posibilidad de introducir bugs. Además, operar a un nivel de abstracción
superior, más cercano al nivel de la semántica del pensamiento del programador,
aumentaría la velocidad a la que puede probar diseños alternativos y así llegar a
uno mejor.

% Por qué refactorings automáticos de alto nivel
% REF a A survey of Software Refactoring
[La referencia] concluye que los refactorings automáticos de alto nivel son más
escalables y performantes. También que los refactorings automáticos aumentan la
velocidad de desarrollo, resultando en una reducción de costos y un aumento de
la calidad del software porque los desarrolladores tienen la posibilidad de
probar más cambios con menos esfuerzo.

Al explorar refactorings de alto nivel surge la pregunta de cuán alto es
convienente subir.
% Los dos objetivos contrapuestos de los refactorings
% REF (Refactoring Object-Oriented Frameworks)
(Refactoring Object-Oriented Frameworks) expone dos objetivos de los
refactorings automáticos que se contraponen. Un objetivo es que el refactoring
sea expresivo, que su nivel sea lo más alto posible para mantener al usuario lo
más lejos posible de toda la complejidad que implica realizar el cambio, y el
otro es que preserve el comportamiento. Cuanto más alto es su nivel más difícil
es justificar que preservan el comportamiento del programa, por lo tanto
queremos que los refactorings sean suficientemente pequeños para poder tener una
buena confianza de que preservan el comportamiento y suficientemente abstractos
para que sean útiles.

% Preguntas que se hace este trabajo, por qué es exploratorio
Este trabajo busca sentar las bases para poder analizar los refactorings
automáticos de alto nivel.



\subsection{Investigaciones más cercanas}
% Preguntas guía
%   - cuáles son los trabajos más parecidos? los de patrones
% REF A Methodology for the Automated Introduction of Design Pattern
Este paper es similar porque busca implementar un refactoring que introduce un
patrón de diseño. El foco está puesto en la identificación de transformaciones
parciales que permitan reestructurar el programa desde un estado precursor
elegido hasta uno con el patrón aplicado. Toman este camino porque su principal
objetivo es automatizar la generación de las transformaciones necesarias. Este
objetivo es atractivo pero no necesariamente valioso. Todavía no contamos con la
información necesaria para saber si ese nivel de abstracción es apropiado para
los desarrolladores y si esas transformaciones son generalizables.

% REF
%   - Automated refactoring to the Strategy design pattern
%   - Automated refactoring of super-class method invocations to the Template Method design pattern
%   - Automated refactoring to the Null Object design pattern
Estos 3 papers implementan la introducción de patrones de diseño de manera
automática en Java. Además, automatizan la identificación de oportunidades para
aplicar el refactoring. La herramienta evalua todo el código, identifica partes
del mismo que se beneficiarían con la aplicación del refactoring y ofrece
aplicarlo. Los dos problemas principales
% REF a programmer-friendly refactoring tools
que se pueden encontrar son que el nivel de automatización no se ajusta a las
prácticas actuales de desarrollo y que las restricciones impuestas al código
precursor para poder aplicar el refactoring son demasiado estrictas, perdiéndose
así un gran número de casos.

%   - qué no cubren esos trabajos?
%   - cuáles son las diferencias con este trabajo?


\section{Valor de realizar este trabajo}

% Cuáles son los objetivos de este trabajo?
El objetivo de este trabajo es implementar refactorings automáticos de alto
nivel para:

\begin{itemize}
    \item Entender mejor sus limitaciones en relación a los atributos en
      contraposición (correlación negativa) expresividad y precisión.
    \item Analizar los desafíos que encontremos para guiar desarrollos futuros.
    \item Contar con una primera herramienta que nos permita medir, analizar y
      evaluar cómo se relaciona el programador con la misma y cómo podría
      mejorarse para maximizar su productividad.
\end{itemize}

% Por qué es necesario realizar una exploración de refactorings automáticos de alto nivel?
Se necesitan implementaciones de refactorings de este tipo para poder evaluar
sus potenciales beneficios y así poder mejorar las herramientas que los
aplican. Las IDEs más utilizadas y poderosas no proveen actualmente refactorings
automáticos de este tipo, los disponibles son más simples. Vemos en esto una
oportunidad de mejora y una necesidad de crecimiento grande.

Orientamos esta herramienta a la automatización de la aplicación del refactoring
porque el principal problema que vemos con las herramientas que automatizan más
etapas es que no requieren intervención del programador. Ésto, como ya vimos, no
se ajustan a su forma de trabajo lo cual lleva a que ese tipo de herramientas
corran dos riesgos:

\begin{description}
    \item[Baja frecuencia de uso] Nadie los usaría ya que asumen que son parte
      de una etapa del desarrollo que no existe.
    \item[Incrementar los costos de un proyecto] El problema es que si el
      refactoring toma decisiones buenas en ciertas dimensiones de calidad, pero
      que perjudican la mantenibilidad del código por hacerlo más complejo o
      menos entendible, como luego el proyecto lo tienen que seguir
      interviniendo programadores humanos cada vez les tomará más tiempo
      realizarle modificaciones; lo cual resultará en un incremento del tiempo
      de desarrollo.
\end{description}

Mientras no haya una manera precisa de expresar comprensibilidad y diseño del
dominio la intervención humana será necesaria y proveerá una reducción de costos
mayor que la utilización de
% REF la survey que decía que el 80% del costo del software es mantenimiento
refactorings completamente automáticos.
