\section{Motivación y smells}
El software evoluciona y con él todos sus componentes. Uno de ellos son los métodos, los cuales a
medida que pasa el tiempo van ganando funcionalidad y cada vez se vuelven más complejos. Los
atributos más tangibles que hacen crecer esa complejidad son:

\begin{itemize}
    \item el largo medido en líneas de código o statements
    \item la cantidad de parámetros que recibe
    \item la cantidad de variables temporales que define
\end{itemize}

% REF Fowler y Beck
Este crecimiento hace que el método sea más grande y reduce su entendibilidad. Al encontrarse con un
método con estas características el programador aplicará un refactoring para poder reducir su
complejidad y entenderlo más fácilmente. Sin volverlo más entendible corre el riesgo de
malinterpretarlo y modificarlo erróneamente o utilizarlo erróneamente, lo cual introduciría un bug
en el programa. Además, aplicar un refactoring le permite correr menos riesgos a él y aumenta la
mantenibilidad del código lo cual implica un gran ahorro de costos.
Existen dos principales razones que reducen la comprensibilidad de un método:

% WRITE puedo poner un ejemplo de cada una de éstas.
\begin{itemize}
    \item tiene más de una responsabilidad
    \item trabaja en más de un nivel de abstracción
\end{itemize}

% WRITE ejemplo de extract method
La principal herramienta para reducir la complejidad producida por esas razones es el refactoring
Extract Method. En el primer caso se delega la responsabilidad a otro método y en el segundo se
realizan las tareas de un nivel de abstracción menor al del método en otro método al que también se
delega. De esta forma el método contará con una sola responsabilidad y la cumplirá realizando
acciones en un solo nivel de abstracción. Así consigue revelar la intención del método y solo
mostrar como se consigue su objetivo delegando a otros métodos. Se agrega un nivel de indirección
por la llamada al método, la delegación, pero se gana en claridad lo cual deja un saldo positivo
para el sistema. Sin embargo, existen casos en los que Extract Method no se puede realizar o las
formas en las que se puede realizar no consiguen un aumento de la entendibilidad del método o una
reducción de su complejidad. En estos casos Extract Method no puede aplicarse porque:

% WRITE mostrar ejemplos
% WRITE acá me podría extender con secciones de código que expliquen los dos casos, el pasar varios
% parámetros y el asignar a más de una variable.
\begin{itemize}
    \item las variables temporales y los parámetros están tan entrelazados en el código que al extraer 
    un método hay que pasarle demasiados parámetros
    \item cualquier sección significativa que se intenta extraer modifica más de una variable que
    luego se sigue utilizando
\end{itemize}

% REF Fowler + Beck
En estos casos la sugerencia es utilizar Extract Method to Method Object.


\section{¿En qué consiste?}

Algunos detalles dependen específicamente del lenguaje de programación utilizado, esta descripción
de alto nivel asumirá un caso estándar en Smalltalk. Los pasos a seguir para extraer el código
del método M a un Method Object son:

\begin{enumerate}
    \item Se crea una clase para el Method Object.
    \item Se le agrega un mensaje de creación de instancia que recibe todos los parámetros y
    variables de instancia utilizadas en M y los asigna a variables de instancia del Method Object.
    \item Se le agrega el mensaje \#value al Method Object.
    \item Se copia el código de M dentro del método \#value del Method Object.
    \item Se reemplazan en \#value los nombres de las variables que fueron pasadas por parámetro
    por el nombre de su variable de instancia correspondiente.
    \item Se reemplaza el cuerpo M por la instanciación del Method Object pasándole todos los
    parámetros y variables de instancia previamente utilizadas.
    \item Se corren los tests para verificar la preservación de comportamiento.
    \item Se aplican los refactorings que antes no se podían sobre el método \#value del Method
    Object.
\end{enumerate}

Veamos los pasos en un caso concreto:

\begin{code}
ExampleClass>>methodToExtract: aParam1 with: aParam2
    | temp1 |

    temp1 := self doSomething: ivar1.

    ^(param1 + param2) > 3 and: [temp1 > 5]
\end{code}

\lstinline{methodToExtract} es el método objeto de nuestro refactoring. Luego de crear la clase del
Method Object le agregaremos el mensaje de creación de instancia:

\begin{code}
MethodObject>>initializeWith: aParam1 with: aParam2 with: anIvar1 with: aSelf
    ivarParam1 := aParam1.
    ivarParam2 := aParam2.
    ivarIvar1 := anIvar1.
    client := aSelf.
\end{code}

Luego agregamos el mensaje \lstinline{value} al Method Object con el contenido de
\lstinline{methodToExtract} y reemplazamos las variables recibidas como parámetro:

\begin{code}
MethodObject>>value
    | temp1 |

    temp1 := client doSomething: ivarIvar1.

    ^(ivarParam1 + ivarParam2) > 3 and: [temp1 > 5]
\end{code}

finalmente reemplazamos el cuerpo del \lstinline{methodToExtract}:

\begin{code}
ExampleClass>>methodToExtract: aParam1 with: aParam2
    ^(MethodObject with: aParam1 with: aParam2 with: ivar1 with: self) value
\end{code}



\section{Descripción completa de la funcionalidad del refactoring implementado del modelo}

% WRITE voy a contar cómo funciona el refactoring de manera top down hasta llegar al detalle del código
% e incluso mostrar partes de código específicas. Todo lo que necesite contar en el camino lo contaré.
% No voy a desarrollar todo, solo lo que crea necesario y dejo señalados los lugares en los que se puede 
% expandir la explicación.

El refactoring implementado tiene como objeto central al encargado de efectivamente aplicar el refactoring,
un objeto llamado \lstinline{ExtractToMethodObject}. Comenzaremos explicando en detalle el comportamiento,
las limitaciones y el funcionamiento de este objeto. Luego pasaremos a ver cómo colabora con el resto del
modelo para ofrecer al usuarion una funcionalidad completa.

A partir de ahora nos referiremos como \lstinline{ExtractToMethodObject} al objeto que aplica el refactoring
especificando en cada caso si hablamos de la clase o una instancia del mismo. \lstinline{ExtractToMethodObject} es
subclase de \lstinline{Refactoring} que tiene al mensaje \lstinline{apply} como único mensaje polimórfico.
% DIAGRAM mostrar un diagrama de clases con refactoring, etmo y el apply
Veamos un ejemplo de utilización del refactoring:

\begin{code}
refactoring := ExtractToMethodObject 
    from: methodToExtract
    toMehtodObjectClassNamed: #MethodObject
    subclassing: Object
    onCategory: 'ThesisExamples'
    withExtractedVariablesToInstanceVariables: variablesNameMapping
    withInstanceCreationMessageFrom: keywordsDefinitions
    evaluatedWith: #value.

refactoring apply.
\end{code}

El mensaje de creación de instancia se encargará de validar todos los colaboradores recibidos para
asegurarse de que la instancia creada será válida y el refactoring podrá ser aplicado. Una vez
instanciado el refactoring, a menos que alguna de las condiciones validadas sea modificada en el
espacio de tiempo entre la creación de la instancia y la aplicación del refactoring, posibilidad que
siempre existe en Smalltalk por ser un entorno de objetos vivos y reflexivo, la aplicación del
refactoring no debería fallar. Es decir, si por alguna razón el refactoring no puede ser aplicado la
responsabilidad de detectarlo es de la clase, no creará instancias que no puedan ser aplicadas ya
que las consideraremos inválidas.

\section{Validaciones}
% WRITE explicar a nivel código que se recibe, qué es parametrizable y qué no
El mensaje de creación de instancia tiene como primer parámetro al método a extraer con el keyword
\lstinline{from:}. Este método a extraer es una instancia de \lstinline{CompiledMethod}, objeto que 
representa un método compilado que la máquina virtual puede interpretar. Este objeto encapsula dos
elementos necesarios del refactoring:

\begin{itemize}
    \item el código del método a extraer y su representación en un AST de objetos a través del mensaje
    \lstinline{methodNode} que nos devuelve una instancia de \lstinline{MethodNode} correspondiente
    al método a extraer.
    \item la clase que es el contexto y para la cual está compilado el método accesible a través
    del mensaje \lstinline{methodClass}, que devuelve una instancia de \lstinline{MethodClass class}
    que es una sublcase de \lstinline{Metaclass}.
\end{itemize}

\subsection*{Validaciones sobre el método a extraer}

\subsubsection*{No puede contener referencias a la pseudovariable \lstinline{super}}

No se permite realizar el refactoring sobre métodos que contienen referencias a \lstinline{super}
porque no se puede replicar el comportamiento de enviar un mensaje a \lstinline{super} en el Method
Object sin modificar considerablemente la clase que contiene el método a extraer, complejizándo el
refactoring demasiado para la utilización que envisionamos por ahora. El receptor de un envío de
mensaje a \lstinline{super} es el mismo que el receptor de un envío de mensaje a la pseudovariable
\lstinline{self}, es decir, la instancia que es el contexto del método que se está ejecutando. La
diferencia reside en que el method lookup inicia en la superclase del receptor, en lugar de
iniciarse en su clase. Por lo tanto, para poder replicar el mismo comportamiento los envíos a
\lstinline{super} deberían seguir realizándose desde la clase del método. Esto se podría conseguir
agregando mensajes a la clase del método que realicen los envíos a \lstinline{super} pero configurar
la creación de estos mensajes para que se realice automáticamente hubiera agregado más pasos a la aplicación
del refactoring y no nos pareció prudente agregarlo sin contar con evidencia de que una versión más simple,
sin esta funcionalidad, fuera aceptada y entendida con facilidad por los usuarios. Veremos un pequeño ejemplo
para ilustrar el caso. Supongamos que el método a extraer es:

\begin{code}
ExampleClass>>methodToExtract
    | temp1 |

    temp1 := ivar1 + super value

    ^temp 1
\end{code}

la forma de replicar el comportamiento sería agregar un mensaje a la clase que realice la llamada a super:

\begin{code}
ExampleClass>>sendToSuper

    ^super value
\end{code}

y utilizar este mensaje desde el método de evaluación del Method Object:

\begin{code}
MehtodObjectClass>>value
    | temp1 |

    temp1 := correspondingIvar + client sendToSuper

    ^temp 1
\end{code}


\subsubsection*{No contiene asignaciones a variables que no sean temporales}

Las variables no temporales son las variables del contexto de la clase:

\begin{itemize}
    \item variables de instancia
    \item variables de clase
    \item variables de pool
\end{itemize}

Estas variables solo son accesibles desde el contexto de la clase, concretamente desde dentro de
un método de la clase. La única forma de asignarles un valor desde fuera de la clase es enviándole
a la clase un mensaje con el valor que queremos asignarles y el método lo asigna, por ejemplo:

\begin{code}
ExampleClass>>>setInstanceVariableTo: aValue

    instanceVariable := aValue.
\end{code}

Estos métodos pueden ser creados automáticamente para replicar el comportamiento de la asignación
desde el Method Object. Sin embargo, como en el caso con las referencias a \lstinline{super} no lo
implementamos porque priorizamos mantener la primer versión del refactoring simple ya que su principal
objetivo es exploratorio. Implementarlo hubiera requerido detectar todas las asignaciones a este tipo 
de variables, ofrecerle la posibilidad al usuario de configurar cómo serán los mensajes para asignarlas
desde el Method Object y luego crearlos automáticamente. El código para detectar las asignaciones
es parte de la validación, si en el futuro quisiera implementarse el flujo completo solo restaría
agregar la parte de configuración para la creación automática de los métodos.



\subsection*{Validaciones sobre los parámetros de la creación de la Method Object Class}

Los siguientes parámetros son los de los keywords \lstinline{toMehtodObjectClassNamed:} que recibe
el nombre de la Method Object class, \lstinline{subclassing:} que recibe la superclase de la Method
Object class y, por último, \lstinline{onCategory:} que recibe la categoría en la cual se ubicará la
Method Object class. Las validaciones sobre estos elementos son las mínimas necesarias para la
definición de una nueva clase, son validaciones que también realiza Cuis cuando intentamos definir
una nueva clase manualmente. Las agregamos aquí también para poder controlar de forma más granular
el feedback que se le da al usuario y los flujos que se siguen. También entra en esta categoría
el selector de evaluación que se recibe en el keyword \lstinline{evaluatedWith:} ya que es un selector
unario que es validado de la misma manera que Cuis.
% WRITE cuando Wilki me responda agregar por qué no puede ser una metaclase
La única validación extra es realizada sobre la superclase, consiste en verificar que no sea una
Meta Clase.



\subsection*{Validaciones sobre las variables de instancia de la clase del Method Object}

El keyword \lstinline{withExtractedVariablesToInstanceVariables:} recibe un parámetro que define
cómo debe llamarse la variable de instancia correspondiente a cada variable a parametrizar. 

\subsubsection*{¿Qué son las variables a parametrizar?}
Las variables a parametrizar son todas aquellas variables referenciadas en el método a extraer que
no son temporales:

\begin{code}
ExampleClass>>methodToExtract: aParam
    | aTemp |

    aTemp := self doSomethingWith: ivar1.
    
    ^aTemp
\end{code}

Este método referencia 4 variables que usaremos como ejemplo de las 4 categorías de variables que podemos
encontrar en un método:

\begin{itemize}
    \item aParam: parámetros del método.
    \item aTemp: las variables temporales del método.
    \item self: las pseudovariables (self y super).
    \item ivar1: las variables del contexto de la clase (variables de instancia, variables de clase y variables de pool)
\end{itemize}

Todas las categorías de variables deben ser parametrizadas excepto las temporales, ya que pertencen
al contexto del método. En el ejemplo anterior el conjunto de variables a parametrizar, es decir que
tenemos que pasarle al Method Object al instanciarlo para que pueda referenciarlas, son: \lstinline{aParam},
\lstinline{self} y \lstinline{ivar1}.

\subsubsection{Continuo hablando sobre las validaciones}

Las variables a parametrizar serán variables de instancia del Method Object, lo cual las hará
disponibles desde cualquier contexto dentro del Method Object y así se podrá descomponer de manera
simple el método extraido. Los nombres son uno de los atributos que más influyen en la
entendibilidad del código y por lo tanto no deben tomarse a la ligera. Los nombres se eligen de
manera contextual, referencian a un objeto por su rol en ese contexto específico. Al cambiar el
contexto, como en este caso que pasan de un método a la clase del Method Object, algunos nombres
deben cambiar. En algunos casos necesitan cambiar por el cambio de contexto pero en otros también
por limitaciones sintácticas como en el caso de las pseudovariables. Si \lstinline{self} es una
variable a parametrizar la variable de instancia correspondiente no puede llamarse también self
porque es un nombre reservado. Las validaciones verifican principalmente que no existan colisiones
de nombres. Los nombres elegidos para las variables de instancia del Method Object no tienen que
pertenecer a ninguna variable de la jerarquía del Method Object, esto incluye variables de
instancia, de clase y de pool de todas las superclases del Method Object. Además, no pueden ser
iguales a los nombres de los parámetros del mensaje de creación de instancia ni a las temporales del
método a extraer. Respetando esa consistencia los nombres pueden ser elegidos libremente. El objeto recibido
es un diccionario que tiene como clave el nombre de la variable a parametrizar y como valor de destino
el nombre que se le debe dar a la variable de instancia correspondiente.



\subsection*{Validaciones sobre las definiciones del mensaje de creación de instancia}

El mensaje de creación de instancia tiene tantos parámetros como variables a parametrizar, por lo tanto
el usuario debe definir cómo se llamará cada keyword y el nombre del parámetro correspondiente. El objeto
que se recibe en el keyword \lstinline{withInstanceCreationMessageFrom:} es una colección ordenada
de objetos que contienen el keyword elegido, el nombre del parámetro que irá en ese keyword y a qué
variable









\begin{itemize}
    \item el método sobre el cual aplicar el refactoring
    \item los valores necesarios para declarar el Method Object
    \begin{itemize}
        \item el nombre que recibirá la clase del Method Object
        \item la superclase del Method Object
        \item la categoría en la cual declarar el Method Object
    \end{itemize}
    \item los nombres de las variables de instancia para las variables parametrizadas
    \item la definición del mensaje de creación de instancia
    \begin{itemize}
        \item las keywords
        \item los nombres de los parámetros
    \end{itemize}
    \item el nombre para el mensaje de evaluación
\end{itemize}


% Quizás estaría bueno contar cómo nos acercamos al problema
%  - se desarrolló con TDD
% WRITE explicar cómo es la colaboración entre el modelo, la UI y el applier


% Precondiciones
% WRITE explicarl el por qué y dar ejemplos concretos
- no pueden haber returns
- no pueden haber referencias a super
- el método no puede contar con asignaciones a variables no temporales

\section{Más allá de la funcionalidad}
% Corte de scope
- agregar un setter de la variable de clase, era demasiado complejo y no era claro cómo presentárselo al usuario.
- solo extrae métodos completos y en general se utiliza así
Creemos que agregar un return explícito y parametrizar self en un caso en el que no es necesario
agregaría complejidad innecesaria al refactoring. Por lo tanto, elegimos preservar el return self
implícito del método. Si encontramos que el método a refactorear cuenta con un return self implícito
no incluimos el return de la evaluación del MethodObject para preservar el return self implícito:
% Decisiones tomadas
- el método tiene un return self implícito

\begin{code}

unaryMessageSelector

 1 + 1.

unaryMessageSelector
 
 (MethodObject new) value.

unaryMessageSelector

 ^(MethodObject with: self) value.

\end{code}

% Decisiones tomadas
- se realiza en el contexto de una clase, si se realizara en un contexto más amplio qué problemas encontraríamos?
- no puede haber shadowing de variables porque dejó de estar permitido

\section{¿Cómo se utiliza? (UI \& UX)}
% Interfaz de usuario (UI & UX)
- qué desaniman a los usuarios de utilizar los refactorings automáticos y qué hicimos al respecto?
    - no saber bien qué hace el refactoring
    explicaciones en texto y visualización de los cambios
    - hasta dónde llegan los cambios
    luego de aplicar el refactoring se muestra qué se identificó como una posible necesidad de cambio manual
- validaciones que impedirían aplicar el refactoring, cómo se muestran? Estas validaciones se
realizan antes de mostrar la UI. Completar la UI, osea configurar el refactoring, en algunos
casos puede ser una tarea que demande cierto tiempo, si ya sabemos que no se puede es preferible
que primero resuelva ese problema o directamente no aplique el refactoring y no pierda tiempo
configurándolo.


\section{Preservación del comportamiento}
% WRITE resumir cómo ganamos confianza y los tests más importantes
Aplicaré los refactorings a distintas partes de Cuis y después le voy a correr los tests.