\section{Introduce Null Object}

% Descripción del refactoring
% historia, quién lo inventó?
% qué code smells motivan el introduce null object
% en qué consiste?
% en el paper del pattern muestran 3 formas de implementarla, está bueno repasarlas
% el chequeo por null de una variable. Reducir la cantidad de condicionales simplifica el código, lo
% vuelve más simple. El chequeo por null estará duplicado en cada utilización de esa variable
% WRITE En el paper de null object hay una buena introducción, si estoy falto de ideas puedo sacar de ahí.

% WRITE Ejemplo concreto
% WRITE Precursor del refactoring en referencia al paper de Methodology

% Limitaciones
- qué pasa si hay asignaciones que no es claro si es un null o no?
no estaría bueno tener un assignOptionalField para que en el contexto de la clase siempre sea NullObject?
- qué pasa si algún método devuelve esa ivar? no tendrá algún null check?
en el paper de null object lo resolvieron agregando un método getReference()
- qué pasa si se usa como colaborador en un envío de mensaje?
- qué pasa si hay chequeos por null que no se pueden refactorear?
se corta el refactoring o se reemplazan por isNull() y se refactorea lo que se puede