\section{Decisiones sobre la implementación}

% Por qué en Smalltalk
Un lenguaje dinámico y puro permitirá iterar más rápidamente por lo tanto es una herramienta ideal
para el prototipado y la naturaleza exploratoria de este trabajo.
Al ser un trabajo explorativo no queríamos invertir en lidiar con casos o dificultades de aplicar
los refactorings a un lenguaje particular, por lo tanto se eligió un lenguaje dinámico y simple para
poder realizar cambios a través de metaprogramación.
% REF (A Refactoring Tool for Smalltalk)
Smalltalk ya ha probado ser una buen campo de experimentación para refactorings (A Refactoring Tool for Smalltalk)
Necesitamos que sea un lenguaje orientado a objetos ya que vamos a estar introduciendo patrones de diseño.
% REF (A Refactoring Tool for Smalltalk)
Smalltalk es reflexivo (A Refactoring Tool for Smalltalk) esto permite utilizar el mismo lenguaje
que se está modificando para modificarlo, no es necesario un metalenguaje intermedio.

% Por qué elegimos Cuis
Tiene una distribución para la Universidad de Buenos Aires.

% Por qué elegimos estos dos refactorings
Existen relaciones entre los refactorings, de generalidad y complejidad. Estos parecen ser el
principio sobre el que el resto se basaría:

\begin{itemize}
    \item introduce null object es el más específico de replaceIfWithPolymorphism e
    introduceSpecialCase
    \item extract to method object creemos que es muy útil y sienta las bases para extract method
\end{itemize}



\section{Los refactorings}

% MOVE a la sección de cada refactoring
\subsection{Descripción del refactoring}
\subsection{Ejemplo concreto}

% Precondiciones
% WRITE explicarl el por qué y dar ejemplos concretos
- no pueden haber returns
- no pueden haber referencias a super

% Limitaciones
- se realiza en el contexto de una clase, si se realizara en un contexto más amplio qué problemas encontraríamos?
- no puede haber shadowing de variables porque dejó de estar permitido
% INO
- por qué se reemplazan todos los ifNilChecks y no está soportada la posibilidad de preservar algunos ifNilChecks? 
en el contexto de una clase no tiene sentido porque suponemos que la variable de instancia no
tomará más el valor nil. Pero si le agregamos un mensaje `isNull` a las clases entonces
podríamos seguir chequándolo. 

% Parametrización
% WRITE explicar a nivel código que se recibe, qué es parametrizable y qué no

% Interfaz de usuario (UI & UX)
- qué desaniman a los usuarios de utilizar los refactorings automáticos y qué hicimos al respecto?
    - no saber bien qué hace el refactoring
    explicaciones en texto y visualización de los cambios
    - hasta dónde llegan los cambios
    luego de aplicar el refactoring se muestra qué se identificó como una posible necesidad de cambio manual
- validaciones que impedirían aplicar el refactoring, cómo se muestran? Estas validaciones se
realizan antes de mostrar la UI. Completar la UI, osea configurar el refactoring, en algunos
casos puede ser una tarea que demande cierto tiempo, si ya sabemos que no se puede es preferible
que primero resuelva ese problema o directamente no aplique el refactoring y no pierda tiempo
configurándolo.



% Límites del scope
% WRITE explicación de lo que describe este párrafo
Varios cambios que formarían parte del refactoring si se realizara manualmente quedaron fuera
del scope porque no vimos fácilmente cómo automatizarlo o por que dadas las limitaciones de
tiempo no entraron dentro de la priorización. Dejamos asentadas acá qué cosas quedaron fuera y
en qué categorías se encuentran. Además, explicamos cómo creemos que podrían resolverse en el
futuro o cómo hacemos nosotros para guiar al usuario en las tareas manuales que debería realizar
luego de aplicar el refactoring.

% WRITE dar ejemplos de todas estos items
En varios casos, se produce una explosión de posibilidades (como en las asignaciones de ivars) y
tomar una de todas sería sobresimplificar el problema y presentárselas al usuario sería
difícil debido a la complejidad que hay que trasmitir.
% INO
- ifNilChecks anidados
- las asignaciones a la variable de instancia en la que se está introduciendo el null object
reemplazamos las asignaciones de nil pero no las de colaboraciones.
% ETMO
- el método no puede contar con asignaciones a variables no temporales
- si hay una asignación a una ivar, por ejemplo, deberíamos preguntar por setters y considerar esos
casos.

% Implementación
% WRITE qué contar sobre la implementación y sobre el diseño?

% Preservación del comportamiento
% WRITE resumir cómo ganamos confianza y los tests más importantes
Aplicaré los refactorings a distintas partes de Cuis y después le voy a correr los tests.


\section{Motivación y smells}

Existen pocos lenguajes de programación orientados a objetos, y de uso extendido ninguno, que no
cuenten con una u otra forma de representar el valor especial null. La utilización de null conlleva
como principal riesgo el enviarle un mensaje o derefenciarlo, dependiendo el lenguaje utilizado, lo
cual resulta en un error en runtime o una terminación abnormal del programa. La forma más simple de
prevenir estos errores es realizar chequeos sobre las variables que se sospecha pueden contener null
para evitar enviarles mensajes. Veamos un ejemplo en la clase \lstinline{Behavior} de Cuis:

\begin{code}
Behavior>>allSuperclasses
	"Answer an OrderedCollection of the receiver's and the receiver's  
	ancestor's superclasses. The first element is the receiver's immediate  
	superclass, followed by its superclass; the last element is Object."
	| temp |
	^ superclass
		ifNil: [OrderedCollection new]
		ifNotNil: [temp := superclass allSuperclasses.
			temp addFirst: superclass.
            temp]
\end{code}

En este caso se verifica si la variable de instancia \lstinline{superclass} es nil para evitar
enviarle el mensaje \lstinline{allSuperclasses} y además por que la ausencia de superclase indica
que estamos en la raíz de la jerarquía.
% REF a Fowler
Si null conlleva tantos riesgos, ¿por qué se utiliza con tanta frecuencia? La utilización de null
puede ser la forma más simple y barata de implementar un modelo en determinado estado o caso. Sin
embargo, si los chequeos por null abundan las causas más comunes son:

\begin{itemize}
    \item que la variable de instancia sea lo que se llama un temporary field, un colaborador que no
      forma parte de la esencia del objeto y por lo tanto no es necesario siempre; en los casos en los
      que no es necesario es nil.
    \item que falte una abstracción en el modelo para representar la ausencia de un tipo de objetos.
\end{itemize}

% REF a fowler y a woolf
Fowler y Woolf describen que la motivación para realizar el refactoring son repetidos chequeos por
nil de un colaborador. Los condicionales reducen la entendibilidad del código y al chequear siempre
la misma condición producen código repetido, lo cual suele aumentar la posibilidad de que el
programador introduzca bugs olvidándose de chequear por nil antes de enviarle un mensaje al
colaborador correspondiente. Una de las principales herramientas del paradigma de objetos para
lidiar con la proliferación de condicionales repetidos en el código es la utilización de
polimorfismo. En lugar de chequear manualmente en qué tipo de situación nos encontramos para decidir
qué hacer, se envía un mensaje polimórfico a un objeto que realizará la tarea correspondiente para
esa situación dependiendo de su tipo. La introducción de un Null Object sigue esta guía. Se
reemplazará la utilización de nil por un Null Object, un objeto que representa la ausencia o la
nada del tipo correspondiente y encapsula el comportamiento necesario en las situaciones en las
cuales hay una ausencia de ese tipo. Al contar con un Null Object ya no es necesario verificar si el
objeto es nil, nunca debería ser nil, y se le puede enviar los mismos mensajes que a un Concrete
Object, ya que son polimórficos, contando con que el Null Object sabrá qué hacer. Además de reducir 
la complejidad del código la introducción del Null Object consigue que las extensiones futuras sean
más seguras, ya que no hay que chequear por nil, y más mantenibles porque si se requiere cierto
comportamiento adicional en un caso en el que el colaborador está ausente el cambio quedará
encapsulado en la clase del Null Object.


% En el paper de null object hay una buena introducción, si estoy falto de ideas puedo sacar de ahí.


\section{¿En qué consiste?}

La esencia del refactoring es el reemplazo de condicionales que chequean si una variable es nil
o no por el envío de un mensaje polimórfico. Conseguir ese objetivo requiere de varios pasos que tienen
varias alternativas de implementación:

\begin{enumerate}
    \item Crear una jerarquía para el Null Object.
    \item Reemplazar las asignaciones de nil por asignaciones del Null Object.
    \item Reemplazar los condicionales que chequean nil por envíos polimórficos al objeto.
    \item Mover el comportamiento requerido para los casos en los que el objeto es nil al Null
    Object.
\end{enumerate}

Llamaremos NullComponent a la clase del Null Object y Component a la clase del objeto que representa
un componente que no es nil. El NullComponent será polimórfico con el Component pero no contiene
estado porque sus métodos no tienen comportamiento, están vacíos, o devuelven alguna constante, en
ocasiones en forma de un Null Object de otro tipo. En un lenguaje dinámico como Smalltalk no es
necesario que exista una relación de herencia entre las clases para que puedan utilizarse
indistintamente como en el caso de lenguajes estáticamente tipados. De todas formas se elige
utilizar una relación de herencia y relacionar al Component y al NullComponent en una misma
jerarquía para dejar la relación entre las dos explícita. La jerarquía que los relaciona puede tomar
varias formas:

% NullObject hierarchy diagram 

La primer y más simple alternativa es la propuesta por Fowler. La principal ventaja de esta
alternativa es que no requiere cambiar la clase Component; su principal desventaja es que si se le
agregan mensajes a la interfaz de Component el NullComponent los heredará y si no se recuerda
reescribirlos en el NullComponent con el comportamiento correcto para el NullComponent el programa
tendrá comportamientos inesperados. La segunda alternativa no tiene la desventaja que vimos en la
primera y es por eso que suele preferirse por sobre ella.

Luego de armar la jerarquía se deberán sobrescribir todos los mensajes de la interfaz de Component
en NullComponent con un comportamiento vacío. Luego de realizar esto se reemplazarán las asignaciones de
nil a la variable de instancia. Por ejemplo:

\begin{code}
Context>>initialize
    component := null.

"Pasará a ser"

Context>>initialize
    component := NullComponent new.
\end{code}

Algunas asignaciones pueden tener como lado derecho el envío de un mensaje el cual puede devolver 
nil. Esos casos deben reemplazarse por una delegación que permita asignar el Null Object:

\begin{code}
Context>>sendTask
    component := self latestComponent.

"Se reemplazará por:"

Context>>assignComponentWith: aComponentOrNull
    component := aComponentOrNull ifNil: [NullComponent new] ifNotNil: [aComponentOrNull].

Context>>sendTask
    self assignComponentWith: (self latestComponent).
\end{code}

Se agrega un mensaje que asignará a la variable de instancia una instancia de NullComponent cuando
el valor que devuelva la colaboración del lado derecho sea nil. Esto debe ser así cuando no tenemos
control sobre esas colaboraciones o queremos encapsular la utilización del NullComponent a cierta parte
del código, si se reemplazan todos los lugares donde se utiliza un nil, en lugar de un Component,
por un NullComponent entonces el mensaje de asignación no debería ser necesario.

Luego de realizar las operaciones anteriores la variable de instancia \lstinline{component} no
debería contener en ningún caso nil, solo instancias de Component o NullComponent. El siguiente y
último paso es reemplazar los condicionales por el envío de mensajes polimórficos y el traslado
del comportamiento de nil al NullComponent:

\begin{code}
Context>>launchRocket
    rocketHeight := component ifNil: [0] ifNotNil: [component ignite].

"Luego de aplicar el refactoring"

NullComponent>>ignite
    ^0

Context>>launchRocket
    rocketHeight := component ignite.
\end{code}







\section{Descripción completa de la funcionalidad del refactoring implementado}

El modelo de IntroduceNullObject es más complejo que el de ExtractToMethodObject. Ambos fueron
desarrollados
% REF a TDD
utilizando TDD y la principal diferencia que se encontró es que al empezar a desarrollar de manera
incremental las precondiciones del refactoring eran complejas y requerían que las estructuras
recibidas por el objeto central del modelo, el encargado de aplicar el refactoring, una instancia de
la clase \lstinline{IntroduceNullObject}, fueran también complejas (colleciones con varios niveles
de profundidad). Resultó conveniente repartir las responsabilidades y encapsular las distintas
estructuras que necesita el refactoring en otros objetos que luego éste recibirá. Las precondiciones
chequeadas por el refactoring se redujeron porque los objetos que recibe son válidos entonces las
validaciones espcíficas de esas estructuras están encapsuladas en esos objetos.

El refactoring se utiliza de la siguiente manera:

\begin{code}
refactoring := IntroduceNullObject
    on: 'ivar1'
    of: aRocket class
    usingAsConcreteClass: ConcreteFuelTank
    usingAsNullClass: NullFuelTank
    for: replacementParameters
refactoring apply.
\end{code}

\lstinline{IntroduceNullObject} es subclase de \lstinline{Refactoring} y por lo tanto sobrescribe el
mensaje \lstinline{apply}. Los colaboradores que recibe en el mensaje de creación de instancia son 
los que necesita para poder aplicar el refactoring.

\subsection*{La variable de instancia y la clase}
En el keyword \lstinline{on:} recibe el nombre de la variable de instancia que se reemplazará por el
Null Object. Hasta ahora el refactoring Introduce Null Object parecía aplicarse a un tipo. El tipo
que utilizamos de referencia era Componente y los cambios, reemplazo de asignaciones de nil o 
chequeos por nil, se realizaban sin un scope determinado, es decir, se aplicaban a lo largo de todo
el sistema. Realizar algo así en un sistema dinámico como Cuis no es posible sin sacrificar
demasiada confianza sobre la preservación del comportamiento del refactoring. El proceso para encontrar
todos los lugares en los que se utiliza una instancia de la clase Component no es claro y, aunque lo
fuera, como los tipos son dinámicos la variable podría contener un conjunto de tipos durante su
vida, no solo Component, dependiendo el momento de ejecución. Por último siempre queda el problema de
la metaprogramación que vuelve inútil cualquier intento de inferencia de tipos estática. Por esta
razón decidimos darle al cambio un scope determinado y que el usuario guíe al refactoring con su
conocimiento del modelo.

El scope elegido es el de una clase. La clase provee un equilibrio entre el método, que sería
demasiado restrictivo, probablemente existan pocos casos en los que se justifique introducir este
refactoring solo para eliminar los condicionales de un solo método, y todo el sistema, ya que no
existe un scope claro más grande que la clase o por lo menos Cuis no lo provee ya que las categorías
o los paquetes no permiten ese tipo de funcionalidad. La clase que se utilizará como contexto es la 
recibida en el keyword \lstinline{of:}.

Dentro de una clase se eligirá un colaborador interno o variable de instancia en particular sobre la
cual se realicen chequeos por nil para aplicar el refactoring. Al no contar con un tipo específico
para buscar todas las variables es necesario fijar la o las variables que el programador, por su
conocimiento del modelo, ya sabe que deberían pertenecer a ese tipo. Esto se apoya en las ventajas
ya expuestas de floss refactoring, el caso en el que el programador tiene claro el cambio que quiere
realizar y la herramienta lo automatiza para reducir la posibilidad de errores y aumentar la
velocidad de ejecución. La herramienta no asume entender el modelo ni poder cambiarlo sin guía del
usuario. Se elige como variable a fijar una variable de instancia porque son las variables que están
disponibles a través de todo el scope de la clase y son más comunes que las variables de clase o
las variables de pool.

El refactoring tendrá como objeto central a la variable de instancia elegida, todo los condicionales
modificados serán sobre esa variable. La única precondición que incluye a la variable de
instancia o a la clase es la que verifica que la variable pertenezca a la clase.

\subsection*{La jerarquía de Null Object}

La jerarquía que se asumirá será aquella en la que Component y NullComponent heredan de
AbstractComponent, veremos por qué más adelante. Sin embargo, no es necesario que efectivamente la
implementen, no es una precondición. La única precondición con respecto a las clases Component y
NullComponent es que no sean iguales. Si se utilizara la misma clase no se podrían enviar mensajes
polimórficos que se respondan de forma distinta dependiendo el tipo de la variable.

\subsection*{Los argumentos para los condicionales a reemplazar}

Los condicionales a reemplazar ya están determinados por la variable de instancia y su clase. Smalltalk no
cuenta con condicionales como parte de su sintaxis, éstos se implementan con polimorfismo. Por lo tanto,
los tipos de condicionales soportados por el refactoring no son más que un conjunto mensajes:

\begin{itemize}
    \item Nil Selectors
    \begin{itemize}
        \item \lstinline{ifNil:}
        \item \lstinline{ifNil:ifNotNil:}
        \item \lstinline{ifNotNil:}
        \item \lstinline{ifNotNil:ifNil:}
    \end{itemize}

    \item Boolean Selectors
    \begin{itemize}
        \item \lstinline{ifTrue:}
        \item \lstinline{ifTrue:ifFalse:}
        \item \lstinline{ifFalse:}
        \item \lstinline{ifFalse:ifTrue:}
    \end{itemize}
\end{itemize}

Los Nil Selectors siempre tienen que tener como receptor del mensaje a la variable de instancia y los
Boolean Selectors tienen que tener como receptor del mensaje a un envío de mensaje \lstinline{isNil} a la
variable de instancia:

\begin{code}
"Nil Selector: ejemplo de receptor"
component ifNil: [^4]    

"Boolean Selector: ejemplo de receptor"
component isNil ifTrue: [^4]    
\end{code}

Todos los condicionales soportados serán reemplazados por el envío de un mensaje polimórfico:

\begin{code}
Context>>fuel
    component ifNil: [0] ifNotNil: [component fuelLeftIn: unit]

"Pasara a ser:"
Context>>fuel
    component polymorphicSelector: units

"Agregando:"
Component>>polymorphicSelector: aUnit
    ^self fuelLeftIn: aUnit

NullComponent>>polymorphicSelector: aUnit
    ^0
\end{code}

El usuario deberá elegir el selector polimórfico y los nombres de los parámetros para las variables
a parametrizar. En el ejemplo anterior el selector polimórfico es \lstinline{polymorphicSelector} y
el nombre del parámetro \lstinline{aUnit}.

% DIAGRAM un diagrama de objetos mostrando cómo intereactúan INO con RINWPP

\section{Funcionalidad de ReplaceIfNilWithPolymorphismParameters}

\lstinline{ReplaceIfNilWithPolymorphismParameters} es una clase que representa los argumentos necesarios
para reemplazar un chequeo por null con polimorfirsmo. Los colaboradores internos de la clase son el
\lstinline{methodNode} del método, la raíz del AST, en el cual se encuentra el chequeo por null,
el \lstinline{messageNode} correspondiente al chequeo por null y una instancia de
\lstinline{VariablesToParametrizeKeywordsDefinitions}. Los dos primeros identifican y referencian al
chequeo por null que reemplazaremos. El tercero es un objeto que contiene la definición del 
mensaje polimórfico: los keywords, qué variable a parametrizar irá en qué keyword y los nombres de
los parámetros.

\lstinline{ReplaceIfNilWithPolymorphismParameters} valida que el \lstinline{messageNode} sea un mensaje
soportado por el refactoring.

% WRITE Explicar cómo se valida que el mensaje sea soportado y las complicaciones que trae eso.
% esto no lo voy a escribir hasta que no haga los tests para ver si ya funciona todo bien o no 
Verificar que el \lstinline{MessageNode} represente un envío de mensaje soportado es un poco más 
complicado que verificar que el selector del nodo pertenezca al conjunto de selectores soportados
y que el receptor del mensaje sea una variable de instancia de la clase del método. La causa son algunas
optimizaciones que se realizan al construir el AST, en específico los MessageNode de un conjunto de mensajes,
dentro de los cuales encontramos a ifNil:ifNotNil: ifNotNil: y ifNotNil:ifNil:.


Además, verifica que la definición del mensaje polimórfico sea consistente con las variables a parametrizar
en los bloques y que no existan colisiones entre los nombres de los parámetros del mensaje polimórfico
y las variables temporales de los bloques que recibe el chequeo por null como argumento.


\subsection{Validaciones de la colección de ReplaceIfNilWithPolymorphismParameters}

% WRITE esto debería cambiar cuando demos la posibilidad de seleccionar un mensaje que ya existe
% para crearlo solo de un lado
\lstinline{IntroduceNullObject} valida que los selectores seleccionados para los mensajes polimórficos
no colisionen con otro mensaje de la jerarquía del Null Object. 


\subsection{Aplicación del refactoring}

Al contar con la colección de \lstinline{ReplaceIfNilWithPolymorphismParameters} la aplicación del
refactoring reemplaza cada uno de esos chequeos de nil con lo argumentos correspondientes. Los
cambios que se deben realizar son:

\begin{itemize}
    \item agregar el mensaje polimórfico al Component
    \item agregar el mensaje polimórfico al NullComponent
    \item reemplazar el chequeo de nil por el envío del mensaje polimórfico
\end{itemize}

El proceso para compilar el nuevo método en las clases Component y NullComponent es igual, lo que 
cambia es el bloque utilizado como fuente:

\begin{code}
component ifNil: ['The NullComponent block'] ifNotNil: ['The Component block']
\end{code}

Los mensajes que tienen un solo keyword no cuentan con un bloque para alguna de las dos clases, en
ese caso el método que se compilará será vacío ya que no se requiere ningún comportamiento si el
objeto tiene ese tipo. Veamos un ejemplo:

\begin{code}
Context>>fuel
    ^component ifNotNil: [
        self fuelLevelChecked.
        component fuelLevel]

Component>>fuelLevelWithLog: context
    context fuelLevelChecked.
    ^self fuelLevel

NullComponent>>fuelLevelWithLog: context
    "Empty method"

Context>>fuel
    component fuelLevelWithLog: self
\end{code}

El chequeo por nil se realizó con el mensaje \lstinline{ifNotNil:}, por lo tanto si el objeto es nil
no se espera comportamiento alguno. Por esa razón el método en la clase \lstinline{NullComponent} resulta
vacío.

La generación del código de los métodos para cada caso se realiza reemplazando todas las variables
a parametrizar, todas las variables que se referencian en el bloque excepto las que sean temporales
% WRITE qué sucedía con los argumentos de los bloques? puede ser que no tenían?
del bloque, por el nombre del parámetros correspondiente. Para realizar esto necesitamos obtener el 
código fuente del bloque y luego obtener el rango en el código fuente del método de cada variable a
parametrizar y restarle el offset del inicio del bloque.

Finalmente, se reemplaza el chequeo de nil por el envío del mensaje polimórfico. Se toma el código
fuente del método, se realiza el reemplazo y luego se vuelve a compilar en la clase Context. El
único detalle de este proceso es que la obtención del rango que comprende el chequeo de nil se debe
realizar manualmente porque el \lstinline{Encoder} no posee esta información. En el caso de
\lstinline{MessageNode}s se cuenta con el rango de los keywords y sus argumentos pero no del
receptor. Por lo tanto el rango de un \lstinline{MessageNode} incluyendo al receptor debe tomar el
lado derecho del rango del envío de mensaje y el lado izquierdo del receptor. El receptor en nuestro
caso puede ser una variable o un envío de mensaje (\lstinline{instVar isNil}) por lo cual es simple
obtener el rango consultando al Encoder. La construcción del envío del mensaje la realiza el
\lstinline{ReplaceIfNilWithPolymorphismParameters} correspondiente, ya que cuenta con el mapeo entre
variable a parametrizar y keyword del selector polimórfico. 


\subsection{¿Cómo se utiliza?}

El usuario puede iniciar el refactoring desde el listado de clases en el System Browser porque lo que
necesita el \lstinline{RefactoringApplier} para iniciar la obtención de los argumentos es la clase
a utilizar como contexto:

% DIAGRAM mostrar cómo se selecciona el refactoring

El \lstinline{RefactoringApplier} es el encargado de recabar los argumentos para el
\lstinline{IntroduceNullObject}, crear el refactoring y aplicarlo. Colaborará con el
\lstinline{IntroduceNullObjectForm} y el \lstinline{IntroduceNullObjectRequest} para conseguir esto.
La distribución de responsabilidades quedará reflejada en el siguiente diagrama de secuencia
que muestra cómo colaborane en un caso exitoso:

% DIAGRAM colaboración en caso exitoso

y la experiencia para el usuario sería:

% DIAGRAM mostrar el paso a paso de aplicación del refactoring.

\subsection{Comentarios adicionales}
% WRITE ¿Por qué lo aplicamos solo en el contexto de una clase? ¿Qué otras alternativas se nos ocurrieron
% en un sistema dinámico como Smalltalk en el que no contamos con el tipo o la clase de una variable?
Ya veremos que es fácil pasar de realizar el cambio en el contexto de una clase a realizarlo en todo
el sistema. Bueno quizás no es tan fácil en un sistema dinámico como Smalltalk. Pero tiene ventajas
y desventajas realizarlo de esta manera, si lo vamos aplicando progresivamente podemos identificar
más fácil si algo se rompe y dónde.


% WRITE Precursor del refactoring en referencia al paper de Methodology

\subsection{Preservación del comportamiento}

Aplicaré los refactorings a distintas partes de Cuis y después le voy a correr los tests.


\section{Motivación y smells}
El software evoluciona y con él todos sus componentes. Uno de ellos son los métodos, los cuales a
medida que pasa el tiempo van ganando funcionalidad y cada vez se vuelven más complejos. Los
atributos más tangibles que hacen crecer esa complejidad son:

\begin{itemize}
    \item el largo medido líneas de código o statements
    \item la cantidad de parámetros que recibe
    \item la cantidad de variables temporales que define
\end{itemize}

% REF Fowler y Beck
Este crecimiento hace que el método sea más grande y reduce su entendibilidad. Al encontrarse con un
método con estas características el programador aplicará un refactoring para poder reducir su
complejidad y entenderlo más fácilmente. Sin volverlo más entendible corre el riesgo de
malinterpretarlo y modificarlo erróneamente o utilizarlo erróneamente, lo cual introduciría un bug
en el programa. Además, aplicar un refactoring le permite correr menos riesgos a él y aumenta la
mantenibilidad del código lo cual implica un gran ahorro de costos.
Existen dos principales razones por las cuales un método se vuelve complejo:

% WRITE puedo poner un ejemplo de cada una de éstas.
\begin{itemize}
    \item tiene más de una responsabilidad
    \item trabaja en más de un nivel de abstracción
\end{itemize}

% WRITE ejemplo de extract method
La principal herramienta para reducir la complejidad producida por esas razones es el refactoring
Extract Method. En el primer caso se delega la responsabilidad a otro método y en el segundo se
realizan las tareas de un nivel de abstracción menor al del método en otro método al que también se
delega. De esta forma el método realizará un solo trabajo a partir de un solo nivel de abstracción
revelando la intención del mismo y no cómo se consigue si no delegando a otros métodos. Se agrega un
nivel de indirección por la llamada al método, la delegación, pero se gana en claridad lo cual deja
un saldo positivo para el sistema. Sin embargo, existen casos en los que Extract Method no se puede
realizar o las formas en las que se puede realizar no consiguen un aumento de la entendibilidad del
método o una reducción de su complejidad:

% WRITE mostrar ejemplos
% WRITE acá me podría extender con secciones de código que expliquen los dos casos, el pasar varios
% parámetros y el asignar a más de una variable.
\begin{itemize}
    \item las variables temporales y los parámetros están tan entrelazados en el código que al extraer 
    un método hay que pasarle demasiados parámetros
    \item cualquier sección significativa que se intenta extraer modifica más de una variable que
    luego se sigue utilizando
\end{itemize}

% REF Fowler + Beck
En estos casos la sugerencia es utilizar Extract Method to Method Object.


\section{¿En qué consiste?}

Algunos detalles dependen específicamente del lenguaje de programación utilizado, esta descripción
de alto nivel asumirá un caso estándar en Smalltalk. Los pasos a seguir para extraer el código
del método M a un Method Object son:

\begin{enumerate}
    \item Se crea una clase para el Method Object.
    \item Se le agrega un mensaje de creación de instancia que recibe todos los parámetros y
    variables de instancia utilizadas en M y los asigna a variables de instancia.
    \item Se le agrega el mensaje \#value.
    \item Se copia el código de M dentro del método \#value del Method Object.
    \item Se reemplazan en \#value los nombres de las variables que fueron pasadas por parámetro
    por el nombre de su variable de instancia correspondiente.
    \item Se reemplaza el cuerpo M por la instanciación del Method Object pasándole todos los
    parámetros y variables de instancia previamente utilizadas.
    \item Se corren los tests para verificar la preservación de comportamiento.
    \item Se aplican los refactorings que antes no se podían sobre el método \#value del Method
    Object.
\end{enumerate}

Veamos los pasos en un caso concreto:

\begin{code}
ExampleClass>>methodToExtract: aParam1 with: aParam2
    | temp1 |

    temp1 := self doSomething: ivar1.

    ^(param1 + param2) > 3 and: [temp1 > 5]
\end{code}

\lstinline{methodToExtract} es el método objeto de nuestro refactoring. Luego de crear la clase del
Method Object le agregaremos el mensaje de creación de instancia:

\begin{code}
MethodObject>>initializeWith: aParam1 with: aParam2 with: anIvar1 with: aSelf
    ivarParam1 := aParam1.
    ivarParam2 := aParam2.
    ivarIvar1 := anIvar1.
    client := client.
\end{code}

Luego agregamos el mensaje \lstinline{value} al Method Object con el contenido de
\lstinline{methodToExtract} y reemplazamos las variables recibidas como parámetro:

\begin{code}
MethodObject>>value
    | temp1 |

    temp1 := client doSomething: ivarIvar1.

    ^(ivarParam1 + ivarParam2) > 3 and: [temp1 > 5]
\end{code}

finalmente reemplazamos el cuerpo del \lstinline{methodToExtract}:

\begin{code}
ExampleClass>>methodToExtract: aParam1 with: aParam2
    ^(MethodObject with: aParam1 with: aParam2 with: ivar1 with: self) value
\end{code}



\section{Descripción completa de la funcionalidad del refactoring implementado del modelo}
% Decisiones tomadas
- el método tiene un return self implícito

\begin{code}

unaryMessageSelector

 1 + 1.

unaryMessageSelector
 
 (MethodObject new) value.

unaryMessageSelector

 ^(MethodObject with: self) value.

\end{code}


% Parametrización
% WRITE explicar a nivel código que se recibe, qué es parametrizable y qué no
El 


% Quizás estaría bueno contar cómo nos acercamos al problema
%  - se desarrolló con TDD
% WRITE explicar cómo es la colaboración entre el modelo, la UI y el applier


% Precondiciones
% WRITE explicarl el por qué y dar ejemplos concretos
- no pueden haber returns
- no pueden haber referencias a super
- el método no puede contar con asignaciones a variables no temporales

\section{Más allá de la funcionalidad}
% Corte de scope
- agregar un setter de la variable de clase, era demasiado complejo y no era claro cómo presentárselo al usuario.
- solo extrae métodos completos y en general se utiliza así
Creemos que agregar un return explícito y parametrizar self en un caso en el que no es necesario
agregaría complejidad innecesaria al refactoring. Por lo tanto, elegimos preservar el return self
implícito del método. Si encontramos que el método a refactorear cuenta con un return self implícito
no incluimos el return de la evaluación del MethodObject para preservar el return self implícito:

% Decisiones tomadas
- se realiza en el contexto de una clase, si se realizara en un contexto más amplio qué problemas encontraríamos?
- no puede haber shadowing de variables porque dejó de estar permitido

\section{¿Cómo se utiliza? (UI & UX)}
% Interfaz de usuario (UI & UX)
- qué desaniman a los usuarios de utilizar los refactorings automáticos y qué hicimos al respecto?
    - no saber bien qué hace el refactoring
    explicaciones en texto y visualización de los cambios
    - hasta dónde llegan los cambios
    luego de aplicar el refactoring se muestra qué se identificó como una posible necesidad de cambio manual
- validaciones que impedirían aplicar el refactoring, cómo se muestran? Estas validaciones se
realizan antes de mostrar la UI. Completar la UI, osea configurar el refactoring, en algunos
casos puede ser una tarea que demande cierto tiempo, si ya sabemos que no se puede es preferible
que primero resuelva ese problema o directamente no aplique el refactoring y no pierda tiempo
configurándolo.


\section{Preservación del comportamiento}
% WRITE resumir cómo ganamos confianza y los tests más importantes
Aplicaré los refactorings a distintas partes de Cuis y después le voy a correr los tests.