\section{Decisiones sobre la implementación}

\subsection*{¿Por qué Smalltalk?}
Necesitamos que sea un lenguaje orientado a objetos ya que vamos a estar introduciendo refactorings
que son propios de el diseño orientado a objetos. Un lenguaje dinámico y puro permitirá iterar más
rápidamente sobre la herramienta, por lo tanto es ideal para la realización de un prototipo y nos
permite explorar con más soltura. Al ser un trabajo exploratorio no queríamos vernos limitados por
las particularidades de un lenguaje de programación o que las mayores dificultades provengan de la
navegación de una sintaxis compleja y sobrecargada. Smalltalk cuenta con una sintaxis simple y al
ser un lenguaje de objetos puro y reflexivo provee la posibilidad de manipular los programas a
través de metaprogramación de una manera simple y poderosa. La reflexividad de Smalltalk además (A
Refactoring Tool for Smalltalk) permite utilizar el mismo lenguaje que se está modificando para
modificarlo, no es necesario un metalenguaje intermedio.
% REF (A Refactoring Tool for Smalltalk)
Smalltalk ya ha probado ser una buen campo de experimentación para refactorings (A Refactoring Tool
for Smalltalk)
% REF (A Refactoring Tool for Smalltalk)


\subsection*{¿Por qué CuisUniversity?}
% REF a cuis y a OpenSmalltalk VM
Cuis es un entorno enfocado en Smalltalk-80 gratuito y Open Source. Está basado en la OpenSmalltalk
% REF a cuisUniversity
VM, como Squeak, y su principal objetivo es mantenerse su base pequeña y simple. CuisUniversity es
un ambiente basado en Cuis creado con el objetivo de enseñar programación orientada a objetos.
Actualmente, es utilizado en la Universidad de Buenos Aires, en la Universidad de Quilmes y en la
Universidad Católica Argentina. Uno de los objetivos de este trabajo es que permitir a los alumnos
experimentar con este tipo de refactorings.


\subsection*{¿Por qué Introduce Null Object y Extract Method To Method Object?}
Luego de analizar varios refactorings de alto nivel, la mayoría variaciones de patrones de diseño,
nos enfocamos en estos dos porque creemos que presentan un buen balance de abstracción y
simplicidad. Son lo suficientemente abstractos para ser considerados de alto nivel, no son
refactorings simples que ya estén presentes en otras IDEs ni manipulan elementos diminutos como un
nombre ya que operan al nivel de diseño del código. Sin embargo, son más simples que otros, lo cual
nos permitirá explorar y aprender para poder sentar la bases hacia refactorings futuros. El
Introduce Null Object es un caso particular de Introduce Special Case y de Replace If With
Polymorphism, por lo tanto consideramos que es un buen precursor de ellos. De la misma forma,
Extract Method To Method Object es un precursor de todos los refactorings que requieren extraer
código y hacer reemplazos contextuales.


\section{Los refactorings}
% Límites del scope
% WRITE explicación de lo que describe este párrafo
Varios cambios que formarían parte del refactoring si se realizara manualmente quedaron fuera
del scope porque no vimos fácilmente cómo automatizarlo o por que dadas las limitaciones de
tiempo no entraron dentro de la priorización. Dejamos asentadas acá qué cosas quedaron fuera y
en qué categorías se encuentran. Además, explicamos cómo creemos que podrían resolverse en el
futuro o cómo hacemos nosotros para guiar al usuario en las tareas manuales que debería realizar
luego de aplicar el refactoring.