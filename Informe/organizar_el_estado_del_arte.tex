% De dónde salieron los refactorings
\subsection{Desarrollo de software evolutivo}
% Los proyectos de software evolucionan
Los refactorings nacen de una característica esencial del software productivo: el cambio.
Lehman, en 1969, fue uno de los primeros en estudiar la evolución y el mantenimiento del software. Sus 
investigaciones resultaron en un cambio de perspectiva sobre el mantenimiento de software que pasó 
a verse como una forma de desarrollo evolutivo. 
% REF cita a la survey de Lientz y Swanson a la que hacen referencia acá
% https://en.wikipedia.org/wiki/Software_maintenance#cite_note-csse.monash.edu.au-4 Lientz, B.P. and
% Swanson, E.B., Software Maintenance Management, A Study Of The Maintenance Of Computer Application
% Software In 487 Data Processing Organizations. Addison-Wesley, Reading MA, 1980.
[Lientz y Swanson] analizó la proporción de costos que representaban los distintos tipos de mantenimientos:

% FORMAT pasarlo a description?
\begin{itemize}
    \item Adaptaciones: modificationes para adaptarse cambios en su entorno (DBMS, OS).
    \item Mejoras: implementación de requerimientos que cambiaron o son nuevos sobre la funcionalidad.
    \item Correcciones: diagnóstico y corrección de errores.
    \item Preventivos: cambios para incrementar la mantenibilidad del software para prevenir problemas.
\end{itemize}

Los primeros dos tipos de actividades de mantenimiento representaban el 75\% de los costos de mantenimiento,
lo cual deja en claro la visión del mantenimiento como un desarrollo evolutivo. Lehman además propuso
que esa evolución incrementa la complejidad de los sistemas. Al incrementarse el número de entidades
que componen un sistema las interacciones que se producen entre ellos puede aumentar exponencialmente
hasta llegar un punto en el cual no es posible entenderlas. En el contexto de un proyecto de software
esto implicaría que el riesgo de romperlo al modificarlo o extenderlo sería demasiado alto. Casos así
% REF crisis del software http://homepages.cs.ncl.ac.uk/brian.randell/NATO/NATOReports/index.html
llevaron a la crisis del software.
No existe una métrica clara de la complejidad de un sistema de software pero se utilizan algunas aproximaciones
% REF cyclomatic complexity Mc Cabe
como la complejidad ciclomática de Mc Cabe. Sin embargo, dado que la complejidad del software está fuertemente
asociada a su mantenibildad, se suele tratar la primera mejorando a la segunda. Es decir, se intenta
limitar el incremento de la complejidad utilizando técnicas que apuntan a preservar la mantenibilidad
del sistema.


\subsection{Calidad y mantenibilidad del software}
La calidad del software en este contexto se refiere a la calidad estructural, los atributos de calidad
no funcionales sobre los que se apoyan los requerimientos funcionales. No existe una métrica aceptada
% REF a cisq
de la calidad ya que puede ser entendida de distintas formas. Sin embargo, CISQ definió un conjunto de
características con las cuales todo software debería contar para aportar valor de negocio:

\begin{itemize}
    \item fiabilidad: la probabilidad de que el sistema falle, su estabilidad. El principal objetivo
    es evitar downtime del sistema.
    \item eficiencia: mide el grado con el cual cumple los requerimientos de los usuarios en
    terminos de tiempo de respuesta.
    \item seguridad: cuantifica el riesgo de que se encuentren vulnerabilidades que dañen al
    negocio.
    \item mantenibilidad: la capacidad del software de cambiar y adaptarse a las necesidades de los
    usuarios y el mercado.
\end{itemize}

La mantenibilidad vemos que es uno de los cuatros componentes de la calidad y, por lo que vimos
antes, presenta una gran oportunidad para reducir los costos de un proyecto de software. 
La mantenibilidad del software suele describirse en función de varios otros atributos de calidad como:

\begin{itemize}
    \item entendibilidad: cuán costoso es entender el sistema. El desarrollador debe entenderlo para
    poder manipularlo de cualquier manera sin romperlo. El tiempo que le toma y el riesgo que tiene
    de romperlo por falta de comprensión son costos afectados por este factor.
    \item cambiabilidad: cuán costoso es cambiar el sistema. En particular, cuán costoso es realizarle
    los cambios que el negocio y los usuarios demandan de él.
    \item reusabilidad: cuán costoso es reusar los componentes del software para proveer una nueva
    funcionalidad.
    \item testabilidad: cuán costoso es crear tests para verificar una funcionalidad.
    \item extensibilidad: cuán costoso es extender el sistema con nuevas funcionalidades. La diferencia
    con cambiabilidad es que en este caso la funcionalidad es nueva y en el otro ya existe pero es
    modificado su comportamiento.
    \item transferibilidad: cuán costoso es transferir el proyecto o parte de él a otro equipo de
    desarrollo.
\end{itemize}

El incremento de la complejidad se produce en gran parte debido a la erosión del diseño, lo cual
decrementa su mantenibilidad. La degradación del diseño se puede ver más claramente a través del
incremento de los costos de cada uno de los componentes de la mantenibilidad de un sistema. Por ejemplo,
un diseño que no previó necesaria la flexibilidad que se requiere para implementar una nueva funcionalidad
incrementará los costos de extensibilidad, el tiempo de desarrollo se verá afectado.

%%%% END %%%

\subsection{Técnicas para preservar la mantenibilidad}

Las decisiones de mantenimiento pueden ser más eficientes aceptando esta realidad del proceso de
desarrollo, su esencia evolutiva. 



% Qué procesos de desarrollo existen? Cascada, ágiles?
Existen varios procesos diseñados para poder desarrollar software que mantenga estos atributos 
de calidad con el objetivo de mantener los costos de evolución bajos para que el software no muera.
Uno de ellos son las metodologías ágiles.

% ¿Por qué la evolución incrementa la complejidad? PERSEPECTIVA SIN MÉTRICAS
La erosión del diseño se produce porque no es posible anticipar los cambios que se le realizarán al
software en el futuro entonces el diseño original no es apropiado para incorporar todos los cambios
que se le realizan luego al sistema. El desarrollador no modifica el diseño antes de extender o
modificar el modelo, entonces las modificaciones que le realiza lo vuelven más complejo. Los
proyectos de sotware industriales tienen restricciones de tiempo ajustadas entonces los desarrolladores
introducen modificaciones de la forma más veloz posible sin tener en cuenta la pérdida de calidad.
% WRITE El incremento de la complejidad reduce su mantenibilidad. ¿Qué consecuencias negativas trae esto?
% ¿Qué técnicas existen para lidiar con este problema?
Existen numerosas técnicas para minimizar o reducir el impacto negativo del cambio en los sistemas
de software. Una metodología general para desarrollar software que se basa en la aceptación del
cambio del mismo es la conocida
% REF agregar referencia a extreme programming
extreme programming, la cual utiliza técnicas como:

\begin{itemize}
    \item scrum
    \item tdd
    \item unit testing
    \item refactoring
    \item pair programming
    \item sprints
    \item etc.
\end{itemize}

% ¿Qué son los refactorings?
El refatoring del software es una de ellas. Se utiliza en técnicas de desarrollo iterativo
incremental, no se han visto integradas en sistemas tradicionales con modelos de desarrollo de cascada lineales.
En el contexto de las metodologías ágiles los refactorings se realizan durante el desarrollo del
software como una parte integral del mismo y no como una etapa separada. El término fue acuñado por
% REF agregar referencia a la tesis de Opdyke
Opdyke en su tesis de doctorado y luego popularizado por Fowler, uno de los mayores evangelistas de
% REF agregar referencia a el libro de Fowler
la metodología extreme programming, en su libro \textit{Refactoring: Improving the design of existing code}.
Un refactoring o reestructuración es una modificación al software que no cambia su funcionalidad. El objetivo
de la misma es mejorar la calidad del sistema modificando su estructura interna para volverlo más mantenible,
entendible o que se adapte mejor a futuros cambios o funcionalidades que haya que agregarle. Por
ejemplo, si se desea agregar una funcionalidad y el diseño no la contempló originalmente se puede
agregar al diseño actual o cambiar el diseño primero para que agregar la funcionalidad sea más
simple y se pueda seguir extendiendo el software en esa dirección más fácilmente.

% Cuál es el objetivo de los refactorings