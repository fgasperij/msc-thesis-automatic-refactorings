\section{Introducción a los refactorings, su historia y razón de ser}

% De dónde salieron los refactorings
\subsection{Desarrollo de software evolutivo}
% Los proyectos de software evolucionan
Los refactorings surgen de una característica esencial del software productivo:
el cambio.  Lehman, en 1969, fue uno de los primeros en estudiar la evolución y
el mantenimiento del software. Sus investigaciones resultaron en un cambio de
perspectiva sobre el mantenimiento de software, que pasó a verse como una forma
de desarrollo evolutivo.
% REF cita a la survey de Lientz y Swanson a la que hacen referencia acá
% https://en.wikipedia.org/wiki/Software_maintenance#cite_note-csse.monash.edu.au-4 Lientz, B.P. and
% Swanson, E.B., Software Maintenance Management, A Study Of The Maintenance Of Computer Application
% Software In 487 Data Processing Organizations. Addison-Wesley, Reading MA, 1980.
[Lientz y Swanson] analizó la proporción de costos que representaban los
distintos tipos de mantenimiento:

\begin{itemize}
    \item Adaptaciones: modificaciones para adaptarse a cambios en su entorno
      (DBMS, OS).
    \item Mejoras: implementación de requerimientos que cambiaron o son nuevos
      sobre la funcionalidad.
    \item Correcciones: diagnóstico y corrección de errores.
    \item Preventivos: cambios para incrementar la mantenibilidad del software
      para prevenir problemas.
\end{itemize}

Los primeros dos tipos de actividades de mantenimiento representaban el 75\% de
los costos de mantenimiento, pero como no son de naturaleza correctiva es más
apropiado incluirlos como parte de un desarrollo progresivo o evolutivolo y no
como mantenimiento. Lehman además propuso que esa evolución incrementa la
complejidad de los sistemas. Al incrementarse el número de entidades que
componen un sistema las interacciones que se producen entre ellos pueden
aumentar exponencialmente hasta llegar a un punto en el cual no es posible
entenderlos. La consecuencia de esto sería que los costos de modificar o
extender el comportamiento de un sistema sin provocar consecuencias inesperadas
como cambiar el comportamiento de otra parte del sistema de una forma que
contradiga los requerimientos funcionales se vuelvan inviables. Son este tipo de
casos los que
% REF crisis del software http://homepages.cs.ncl.ac.uk/brian.randell/NATO/NATOReports/index.html
llevaron a la crisis del software.

Si bien no existe una métrica clara que mida la complejidad de un sistema de
software, se utilizan algunas aproximaciones
% REF cyclomatic complexity Mc Cabe
como la complejidad ciclomática de Mc Cabe. Sin embargo, dado que la complejidad
del software está fuertemente asociada a su mantenibildad, se suele tratar la
primera mejorando a la segunda. Es decir, se intenta limitar el incremento de la
complejidad utilizando técnicas que apuntan a preservar la mantenibilidad del
sistema.


\subsection{Calidad y mantenibilidad del software}
La calidad del software en este contexto se refiere a la calidad estructural,
los atributos de calidad no funcionales sobre los que se apoyan los
requerimientos funcionales. No existe una métrica aceptada
% REF a cisq
de la calidad ya que esta puede ser entendida de distintas formas. Sin embargo, CISQ
definió un conjunto de características con las cuales todo software debería
contar para aportar valor de negocio:

\begin{itemize}
    \item fiabilidad: la probabilidad de que el sistema falle, su
      estabilidad. El principal objetivo es evitar downtime del sistema.
    \item eficiencia: mide el grado con el cual cumple los requerimientos de los
      usuarios en términos de tiempo de respuesta.
    \item seguridad: cuantifica el riesgo de que se encuentren vulnerabilidades
      que dañen al negocio.
    \item mantenibilidad: la capacidad del software de cambiar y adaptarse a las
      necesidades de los usuarios y el mercado.
\end{itemize}

Entonces, se puede apreciar que la mantenibilidad es uno de los cuatros
componentes de la calidad y, como se mencionó anteriormente, la misma ofrece una
gran oportunidad para reducir los costos de un proyecto de software.  La
mantenibilidad del software suele describirse en función de varios otros
atributos de calidad como:

\begin{itemize}
    \item entendibilidad: cuán costoso es entender el sistema. El desarrollador
      necesita entenderlo para poder modificarlo o extenderlo sin cambiar su
      comportamiento de maneras inesperadas. El tiempo que le toma y el riesgo
      que tiene de que el resultado viole los requerimientos funcionales por
      falta de comprensión dependen directamente de este atributo.
    \item cambiabilidad: cuán costoso es cambiar el sistema. En particular, cuán
      costoso es realizarle los cambios que el negocio y los usuarios demandan
      de él.
    \item reusabilidad: cuán costoso es reusar los componentes del software para
      proveer una nueva funcionalidad.
    \item testabilidad: cuán costoso es crear tests para verificar una
      funcionalidad.
    \item extensibilidad: cuán costoso es extender el sistema con nuevas
      funcionalidades. La diferencia entre este atributo de calidad y la
      cambiabilidad radica en que en este caso la funcionalidad es nueva y en el
      otro ya existe pero su comportamiento es modificado.
    \item transferibilidad: cuán costoso es transferir el proyecto o parte de él
      a otro equipo de desarrollo.
\end{itemize}

Éstos atributos no son independientes, están relacionados y se afectan los unos
a los otros. El incremento de la complejidad se produce en gran parte debido a
% REF Design Erosion: Problems and causes - https://dl.acm.org/doi/10.1016/S0164-1212(01)00152-2
la erosión del diseño, lo cual decrementa su mantenibilidad. La degradación del
diseño se puede ver más claramente a través del incremento de los costos de cada
uno de los componentes de la mantenibilidad de un sistema. Por ejemplo, un
diseño que no previó necesaria la flexibilidad que se requiere para implementar
una nueva funcionalidad incrementará los costos de extensibilidad y por lo tanto
el tiempo de desarrollo se verá afectado.


\subsection{Técnicas para preservar la mantenibilidad}
Las decisiones de mantenimiento pueden ser más eficientes aceptando esta
realidad del proceso de desarrollo, su esencia evolutiva. Existen varias
metodologías de desarrollo de software con enfoques distintos y prioridades
diferentes. Las metodologías ágiles engloban a varios frameworks de desarrollo
que se contraponen a los procesos pesados más tradicionales y proponen
alternativas más livianas en los cuales prima la aceptación del cambio como guía
del desarrollo. Algunos ejemplos son:

% REF a los frameworks
\begin{itemize}
    \item Extreme Programming (XP)
    \item Kanban
    \item Scrum
\end{itemize}

Los proyectos de software industriales tienen restricciones de tiempo ajustadas,
motivo por el cual los desarrolladores introducen modificaciones de la forma más veloz
posible, sin tener en cuenta la pérdida de calidad. El desarrollador no modifica
el diseño antes de extender o modificar el modelo, entonces las modificaciones
que le realiza al modelo lo vuelven más complejo. La erosión del diseño se
produce porque no es posible anticipar los cambios que se le realizarán al
software en el futuro, entonces el diseño original no es apropiado para
incorporar todos los cambios que se le realizan luego al sistema.


\subsection{Los refactorings}
Un refactoring o reestructuración es una modificación al
% Cuál es el objetivo de los refactorings
software que no cambia su funcionalidad. El objetivo de la misma es mejorar la
calidad del sistema modificando su estructura interna para volverlo más
mantenible, entendible o que se adapte mejor a futuros cambios o funcionalidades
que haya que agregarle. Por ejemplo, si se desea agregar una funcionalidad y el
diseño no la contempló originalmente se puede agregar al diseño actual o cambiar
el diseño primero para que agregar la funcionalidad sea más simple y se pueda
seguir extendiendo el software en esa dirección más fácilmente.

Los refactorings son una de las prácticas del proceso continuo que propone XP
para evitar la erosión del diseño. Se utiliza en varios frameworks de desarrollo
iterativo incremental, pero no se ha visto integrada en sistemas tradicionales
con modelos de desarrollo de cascada lineales. Además, es una parte integral del
desarrollo en el contexto de las metodologías ágiles.

% ¿Qué son los refactorings?
% REF agregar referencia a la tesis de Opdyke
El término fue acuñado por Opdyke en su tesis de doctorado y luego popularizado
por Fowler, uno de
% REF agregar referencia a el libro de Fowler
los mayores abanderados de la metodología extreme programming, en su libro
\textit{Refactoring: Improving the design of existing code}.


\section{La investigación sobre refactorings}

\subsection{¿Son útiles los refactorings?}
Los refactorings son útiles porque permiten preservar la calidad del software,
principalmente su mantenibildad a través de mejoras en:

\begin{itemize}
    \item extensibilidad
    \item comprensiblidad
    \item cambiabilidad
    \item reusabilidad
\end{itemize}

% REF agregar al estudio empírico (Empirical study on the impact...)
Esto ha sido mostrado repetidas veces en estudios empíricos. Esto se debe a que
previene la erosión del diseño ralentizando el incremento de la complejidad del
sistema.  Los cambios que se introducen en el software no fueron previstos por
el diseñador original del sistema, por lo tanto si no se modifica el diseño su
calidad se verá perjudicada. Los refactorings modifican el diseño antes de
agregar las nuevas funcionalidades para que el diseño pueda adoptar los nuevos
cambios sin perder calidad. Esto permite preservar la mantenibilidad del sistema
resultando en una reducción de costos significativa para el proyecto.


\subsection{Áreas principales}
Los software refactorings son investigados desde numerosas perspectivas. Algunos
de los temas más investigados son:

% REF a Trends, opportunities...
\begin{itemize}
    \item refactoring tools: ¿Qué factores afectan el uso de refactorings
      automáticos? ¿Qué refactorings automáticos usan los programadores? ¿Cuáles
      no y por qué? ¿Cómo mejorar las herramientas actuales?
    \item bad smells: la relación entre los bad smells y los refactorings que
      podrían ayudar a lidiar con ellos. No hay estudios empíricos que lleguen a
      conclusiones claras sobre si los code smells son útiles o no para los
      programadores; si ayudan a determinar el lugar en el que un refactoring es
      necesario o qué refactoring es necesario.
    \item refactorings en artifacts que no son código: especificación de
      requerimientos, diseño de más alto nivel, documentación, etc.
    \item patrones de diseño: su utilidad, formas de automatizarlos, la pregunta
      de si el nivel de automatización de los mismos afecta su adopción.
    \item TDD: el refactoring es una importante etapa de esta técnica.
    \item métricas del software sobre los atributos de calidad para medir el
      impacto de los refactorings
\end{itemize}


\subsection{Complejidades de estudiar refactorings}
Los principales desafíos que presenta la investigación de refactorings son:

\begin{itemize}
    % REF agregar al paper de las taxonomías
    \item Comparar y evaluar las herramientas y los refactorings: existen varias
      propuestas de taxonomías cuyo objetivo es proveer un marco para comparar y
      evaluar las herramientas para realizar refactorings y los refactorings en
      sí mismos, pero ninguna que se haya aceptado por completo.
    \item Determinar la preservación del comportamiento: la mayoría de los
      estudios utilizan técnicas semi formales.
      % REF agregar como referencia el estudio que analizaba a los programadores mientras programaban
    \item Entender cómo realizan refactorings los desarrolladores: los estudios
      más precisos son poco representativos y los más generales son poco
      precisos. Sin una clara comprensión de este fenómeno la mayor parte de las
      herramientas basan sus decisiones en hipótesis frágiles.
    \item Métricas para evaluar la calidad del software: estas métricas
      permitirían determinar si un refactoring aumentó la calidad del software o
      redujo la complejidad total del sistema. Existen varios intentos por
      definir una, pero ninguno concluyente. Todavía en general se validan los
      resultados de los refactorings mediante la evaluación de expertos
      ingenieros de software.
\end{itemize}


\section{Refactorings automáticos}
Extreme Programming acepta que el software evoluciona y el diseño cambia
constantemente. La adaptación del software requiere una inversión constante de
energía de parte del desarrollador y es necesaria para que la complejidad no
crezca demasiado. Cambiar el diseño aplicando refactorings para que la
complejidad crezca lo menos posible solo será posible minimizando la energía que
tiene que invertir el
% REF (Practical Analysis for Refactoring)
desarrollador y esto se puede conseguir automatizando los refactorings.

La automatización de los refactorings nos enfrenta a varios interrogantes como
cuándo realizar un refactoring, dónde realizarlo y qué cambiar; estos problemas
se encuentran íntimamente relacionados. A priori, se puede pensar que todas esas
decisiones, al ser etapas de la actividad de refactoring, son buenas candidatas
para la automatización. De hecho, existen investigaciones que intentan procesar
el código y realizarle cambios sin intervención del desarrollador.
% REF (Automated refactoring of super-class method invocations to the Template Method design pattern)
Los principales caminos que se están investigando actualmente para la
automatización de la identificación de diseños pobres y sus correspondientes
refactorings sin intervención del desarrolladors son:

\begin{itemize}
    \item métodos basados en métricas: áreas del código con baja calidad son
      identificadas detectando los mínimos de alguna métrica de calidad.
    \item métodos basados en lógica: el código es traducido a un lenguaje lógico
      intermedio que es analizado con reglas que verifican la calidad de las
      relaciones para identificar defectos.
    \item métodos basado en búsqueda: la mejora del diseño se traduce a un
      problema de optimización de una función de
      % REF función de fitness
      fitness cuyo espacio de búsqueda son los diseños alternativos.
    \item técnicas de visualización: diferentes formas de visualizar el código
      que buscan ayudar al desarrollador a ganar nuevas perspectivas del
      código que le permitan identificar defectos más fácilmente.
\end{itemize}

Sin embargo, la mayoría de las herramientas utilizadas en la industria
automatizan la ejecución del cambio, el desarrollador elige cuándo, dónde y qué
cambio se realizará.
% REF (Trends...)
Un área de investigación que se ha visto relegada es la de code smells. Los code
smells son ciertas características del código que suelen ser síntomas de
problemas más profundos del diseño del software. Esos problemas son buenos
candidatos para ser sometidos a refactorings ya que su adaptación suele
contribuir de manera significativa a la mantenibildad del sistema. La detección
de los mismos y la elección de un refactoring para remediarlo ha recibido poca
atención.

% REF (A Refactoring Tool for Smalltalk)
Las modificaciones que se le realizan a un programa pueden ser divididas en dos
etapas:

\begin{itemize}
    \item refactorings para incluir o modificar funcionalidad sin perjudicar a
      la mantenibilidad
    \item las modificaciones o extensiones
\end{itemize}

Si se cuenta con refactorings automáticos que preservan el comportamiento para
realizar las modificaciones que no cambian el comportamiento entonces, la única
fuente de errores al introducir cambios en un programa son las modificaciones que sí
alteran el comportamiento o lo extienden.
Esto reduce la posible cantidad de errores agilizando el mantenimiento del
software. Realizar
% REF (Programmer-Friendly...)
refactorings manuales conlleva sus propios riesgos, los más frecuentes son:

\begin{itemize}
    \item introducir bugs
    \item consumir más tiempo del disponible
\end{itemize}

y los refactorings automáticos mitigan ambas. Sin embargo, la forma de automatizar los
refactorings no es única. Existen numerosas formas de automatizar el mismo
refactoring. Se pueden categorizar los métodos para la realización de
refactorings automáticos considerando las siguientes dimensiones:

\begin{itemize}
    \item método de aplicación: ¿cuán automático es? ¿automatiza la
      identificación del lugar donde aplicar el refactoring? ¿elige los nombres
      de las nuevas entidades que haya que crear? ¿lo aplica automáticamente?
      ¿cuándo?
    \item preservación del comportamiento: manual, semi-formal, formal.
    \item composición de los refactorings: dinámica o estática. La cantidad de
      refactorings posibles es grande por lo cual se estima que sería útil
      contar con una herramienta que le permita al desarrollador crear sus
      propios refactorings automatizados y luego utilizarlos.
\end{itemize}



\section{Refactorings automáticos de alto nivel}
% Qué son los refactorings de alto nivel?
Los refactorings de alto nivel reciben su nombre por el nivel de abstracción al
cual operan. Éstos suelen ser combinaciones comunes de refactorings más simples,
o de bajo nivel, que pueden o no tener una semántica clara al nivel del diseño
del sistema. Los refactorings de bajo nivel manipulan el código en un contexto
más reducido. Ejemplos de refactorings de bajo nivel son:

\begin{itemize}
    \item renombrar variables o métodos
    \item nombrar constantes
    \item extraer código a un método
    \item inline de un método
    \item cambiar la aridad de un método
\end{itemize}

Los de alto nivel son menos específicos pero tienen un alcance mayor:

\begin{itemize}
    \item introducción de un patrón de diseño
    \item división de una clase en dos que colaboran
    \item cambios a una jerarquía de clases del modelo
\end{itemize}

Estos refactorings realizan cambios que suelen tener una semántica en el nivel
de diseño del modelo del sistema.

% Intro sobre motivación de hacer de más alto nivel, ¿garpa o no garpa?
Las investigaciones todavía no determinan si es mejor que los refactorings sean
de más alto nivel.  Modificar el código es una operación delicada, cuanto más se
automaticen los cambios que debe realizar el desarrollador menor intervención
humana y por lo tanto menor espacio para el error. Sin embargo, estudios acerca
de la utilización de estas herramientas no arrojan resultados claros que
indiquen que los desarrolladores las utilicen con la frecuencia
esperada. Incluso la correlación entre la complejidad del refactoring (cuán alto
es su nivel) y la frecuencia de su utilización se ha
% REF programmer friendly y fitness for purpose
mostrado inversa. Hay estudios que intentan entender a qué se debe esto y cómo
construir herramientas que automaticen los cambios y que los desarrolladores
utilicen.

% Argumentos a favor de alto nivel
% REF A discussion of software..) 
(A discussion of software...) sostiene que los refactorings deben ser más
complejos para poder ayudar al desarrollo de proyectos grandes, es decir que los
refactorings simple no escalan para ser realmente útiles en el contexto de
proyectos de mayor envergadura. Por lo tanto, las herramientas de refactoring
deberían permitir al desarrollador componer refactorings para poder construir
versiones más complejas de los mismos. La posibilidad de componer refactorings
le proveería a las herramientas la escalabilidad necesaria.

A la hora de decidir cómo implemtar y proveer refactorings de alto nivel se
dividen dos vertientes:

\begin{itemize}
    \item compuestos: proveer refactorings simples que se compongan bien o
      proveer una buena forma de componerlos.
    \item compactos: proveer refactorings cuya unidad sea un cambio que tengan
      semántica a nivel diseño.
\end{itemize}

% Cómo implementar esos refactorings de alto nivel
% REF a refactoring tool... a compositional paradigm
(A Refactoring Tool... A Compositional Paradigm) sostiene que los refactorings
complejos deberían realizarse componiendo refactorings más simples. Sin embargo,
algunos refactorings se realizan comúnmente y es tedioso realizarlos incluso con
refactorings automáticos más simples porque no es fácil encontrar refactorings
intermedios que simplifiquen la tarea.

% Características de los refactorings de alto nivel
% REF A Methodology for the Automated Introduction of Design Patterns
(A Methodology for the Automated Introduction of Design Pattern) muestra que los
refactorings de alto nivel tienen más puntos de partida posibles y más destinos
posibles ya que, a diferencia de refactorings más primitivos, están definidos de
forma más relajada. Se puede decir que cuanto más alto el nivel de un
refactoring menos específica es su definición. Los distintos puntos de partida
posibles son:

\begin{itemize}
    \item hoja en blanco: las entidades que se relacionarían en el patrón de
      diseño no se conocen todavía. Este caso no ocurre en la práctica
      usualmente.
    \item anti-patrón: este caso se debe a falta de conocimiento del
      programador. Se soluciona con educación, los posibles malos diseños son
      demasiados para considerarlos uno por uno.
    \item precursor: es un buen diseño para un caso más simple pero que ante
      nuevas necesidades de extensión debe cambiarse.
\end{itemize}



\section{Introducción de patrones de diseño}
% Qué son los patrones de diseño
Los patrones de diseño son soluciones a problemas de diseño que surgen
frecuentemente al construir sistemas con lenguajes orientados a objetos. Los
patrones de diseño son soluciones definidas de una forma que abstrae los
detalles de cada situación pero preserva las fuerzas en contraposición que la
solución pretende balancear. Están definidos al nivel del diseño de un sistema y
contribuyen a que el mismo pueda soportar cierta funcionalidad con más
calidad. Su presencia es común y es por eso que
% Por qué se intentan automatizar
los refactorings de alto nivel automáticos intentan aplicarlos. Al automatizar
refactorings se quiere automatizar lo más posible para ahorrar la mayor cantidad
de energía y tiempo del desarrollador. Además, se busca el nivel más expresivo
posible para que la aplicación del refactoring esté lo más cerca del nivel de
abstracción al que está pensando el desarrollador el cambio que quiere
realizar. Las fuerzas que se contraponen son expresividad y precisión. Cuanto
más alto el nivel de abstracción, más difícil es precisar a nivel de código en
qué consiste el cambio. Los patrones de diseño presentan un balance atractivo
porque son cambios semántico suficientemente específicos para precisarlos en el
código y además están cerca de la forma que tiene el desarrollador de pensar
su cambio.
% REF (Practical Analysis for refactoring)
(Practical Analysis for refactoring) sostiene la aplicación automática de
patrones de diseño permitiría reducir significativamente la energía que necesita
el desarrollador para aplicarlos, lo cual le permitiría explorar más opciones de
diseño con un costo menor.

% REF null object pattern paper
(null object pattern paper) exploró la introducción automática de patrones de
diseño orientada a root canal
% REF root canal refactoring programmer friendly
refactoring. Las herramientas que crearon analizan todo el código en batch,
presentan los candidatos identificados y proveen la opción de aplicar el
refactoring. Como esas existen más investigaciones que exploran la introducción
automática de patrones de diseño, se puede clasificar a las mismas según el tipo
de patrones que analizaron:

\begin{itemize}
    \item estructurales (Abstract Factory y Composite)
    \item de comportamiento (Decorator, Template Method, Null Object y
      State/Strategy)
\end{itemize}

los métodos que utilizan para la identificación de oportunidades de mejora al
diseño y la aplicación de los respectivos refactorings también es variada. Sin
embargo, el principal problema que le vemos a estos trabajos es que no se
ajustan a la forma de trabajo del programador.



\section{Preservación del comportamiento}
% Qué significa preservar el comportamiento
La preservación del comportamiento está presente en la misma definición de un
refactoring y es de máxima importancia. En general, el comportamiento de un
programa suele describirse como una función que va de el conjunto de todos los
posibles inputs al conjunto de todos los posibles outputs. Una reestructuración
del mismo preserva su comportamiento si para todo input el output es el mismo
que antes de la aplicación del refactoring. Esta definición no es útil para
implementarla en las herramientas de automatización.
% Por qué es difícil asegurar la preservación
Además, una dificultad adicional a la hora de formalizar la preservación de la
funcionalidad de un programa es que existen ciertos tipos de software para los
cuales preservar el comportamiento implica más que preservar su funcionalidad,
por ejemplo:

\begin{itemize}
    \item tiempo de ejecución (sistemas de tiempo real)
    \item memoria utilizada y consumo de energía (sistemas embebidos)
    \item condiciones de seguridad (sistemas en los cuales la seguridad es
      crítica)
\end{itemize}

Por esta razón un testing sistematizado y ajustado a los requerimientos de cada
sistema particular es la mejor herramienta con la que se cuenta actualmente.
% REF (Improving refactoring tools in Smalltalk using static type inference)
(Improving refactoring tools in Smalltalk using static type inference) dice que
la única forma de asegurar que los refactorings son correctos es con pruebas
formales pero que las herramientas actuales no realizan esto porque la
complejidad del software actual vuelve demasiado costoso aplicar modelos de
verificación formal a los programas que nos interesa reestructurar.

% Las limitaciones particulares de Smalltalk
% REF (A Refactoring Tool for Smalltalk)
(A Refactoring Tool for Smalltalk) muestra que no es posible asegurar que se
preserva el comportamiento en un lenguaje con tipado dinámico como Smalltalk,
que la única forma de converger a un programa correcto es a través del análisis
dinámico. Incluso el análisis dinámico realizado por el trabajo anterior se basa
en una suite de tests representativa.

% Técnicas para mostrar la preservación del comportamiento
% REF (Automated Application of Design Patterns: A Refactoring Approach)
(Automated Application of Design Patterns: A Refactoring Approach) clasifica las
técnicas que se utilizan para lidiar con la preservación de comportamiento en:

\begin{itemize}
    \item informales: la verificación consiste en las experiencia del
      desarrollador.
    \item semiformales: se describen con lógica las precondiciones y
      poscondiciones y se muestra por qué se cree que preservan el
      comportamiento. Sirve como referencia de lo pensado, para ganar confianza
      en el trabajo realizado y si en el futuro surgiera algún error puede
      corregirse en las descripciones logrando acumular el conocmiento.
    \item formales: verificaciones formales que demuestran la preservación del
      comportamiento.
\end{itemize}

Es poco frecuente la utilización de verificaciones formales porque pocos
lenguajes de programación ampliamente utlizados tienen una semántica formal y un
compilador que la verifica. Además, la complejidad de las demostraciones de
preservación de comportamiento para transformaciones no triviales es intratable.



\section{Herramientas actuales de uso popular}
Esta sección busca mostrar brevemente cuáles son las herramientas de
refactorings automáticos de utilización más generalizada, qué tipos de
refactorings tienen y con qué alcance. Las herramientas que realizan
refactorings automáticos se encuentran como parte de una IDE o como un plugin de
la
% REF survey de stackoverflow de qué IDEs usa la gente
misma. La enumeración no es exhaustiva pero creemos que sí es
representativa. Todas las IDEs proveen algún tipo de refactoring automático
simple como extract method o renombre de entidades. Detallamos a continuación el
soporte presentado para los refactorings extract method to method object e
introduce null object que trataremos en este trabajo:

% REF a cada uno de los sitios de las IDEs
\begin{itemize}
    \item Visual Studio Code: no presenta soporte para ninguno.
    \item Visual Studio: solo realiza extract method y el nuevo método tiene que
      pertenecer a la misma clase. Si necesita devolver más de un resultado
      utiliza los parámetros de salida de C\#.
    \item IntelliJ: provee soporte solo para extract method to method object con
      un scope limitado.  % REF https://www.jetbrains.com/help/idea/extract-into-class-refactorings.html#extract_method_object
      No parametriza el contexto de la clase, solo el contexto local del
      método. La extracción se realiza a una clase interna de la clase base, lo
      cual simplifica bastante el refactoring porque compartimos el scope de
      variables de instancia, de clase, etc.
    \item Eclipse: solo realiza extract method.
    \item XCode: extract to method object.
    \item NetBeans: no presenta soporte para ninguno.
\end{itemize}
